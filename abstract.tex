\cleardoublepage
\thispagestyle{plain}

\pdfbookmark{Kurzfassung}{kurzfassung}
\paragraph{Kurzfassung}
Weihnachten, auch Weihnacht, Christfest oder Heiliger Christ genannt, ist das Fest der Geburt Jesu Christi. Festtag ist der 25. Dezember, der Christtag, auch Hochfest der Geburt des Herrn, dessen Feierlichkeiten am Vorabend, dem Heiligen Abend (auch Heiligabend, Heilige Nacht, Christnacht, Weihnachtsabend), beginnen. Er ist in vielen Staaten ein gesetzlicher Feiertag. In Deutschland, Österreich, der Schweiz und vielen anderen Ländern kommt als zweiter Weihnachtsfeiertag der 26. Dezember hinzu, der auch als Stephanstag begangen wird.

\cleardoublepage
\thispagestyle{plain}



\pdfbookmark{Abstract}{abstract}
\paragraph{Abstract} Christmas or Christmas Day is an annual festival commemorating the birth of Jesus Christ, observed most commonly on December 25 as a religious and cultural celebration among billions of people around the world. A feast central to the Christian liturgical year, it is prepared for by the season of Advent or the Nativity Fast and initiates the season of Christmastide, which historically in the West lasts twelve days and culminates on Twelfth Night; in some traditions, Christmastide includes an Octave. Christmas Day is a public holiday in many of the world's nations, is celebrated culturally by a large number of non-Christian people, and is an integral part of the holiday season, while some Christian groups reject the celebration. In several countries, celebrating Christmas Eve on December 24 has the main focus rather than December 25, with gift-giving and sharing a traditional meal with the family.
