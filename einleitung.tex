%!TEX root = thesis.tex

\chapter{Einleitung}
Weihnachten ist mit Ostern und Pfingsten eines der drei Hauptfeste des Kirchenjahres. Die weihnachtliche Festzeit beginnt liturgisch mit der ersten Vesper von Weihnachten am Heiligabend (siehe dazu auch Christvesper) und endet in den evangelischen Kirchen mit Epiphanias, in der ordentlichen Form des römischen Ritus der katholischen Kirche mit dem Fest Taufe des Herrn am Sonntag nach Erscheinung des Herrn. In der Altkatholischen Kirche und der außerordentlichen Form des römischen Ritus endet die Weihnachtszeit mit dem Ritus der Krippenschließung am Fest der Darstellung des Herrn am 2. Februar, umgangssprachlich Mariä Lichtmess oder auch nur Lichtmess genannt. Der erste liturgische Höhepunkt der Weihnachtszeit ist die Mitternachtsmesse in der Nacht vom 24. auf den 25. Dezember (siehe Christmette).

Als kirchlicher Feiertag ist der 25. Dezember erst seit 336 in Rom belegt. Wie es zu diesem Datum kam, ist umstritten. Diskutiert wird eine Beeinflussung durch den römischen Sonnenkult: Kaiser Aurelian hatte den 25. Dezember im Jahr 274 als reichsweiten Festtag für Sol Invictus festgelegt; zwischen diesem Sonnengott und „Christus, der wahren Sonne“ (Christus verus Sol) zogen die Christen früh Parallelen.

Christen und Nichtchristen feiern Weihnachten heute meist als Familienfest mit gegenseitigem Beschenken; dieser Brauch wurde seit 1535 von Martin Luther als Alternative zur bisherigen Geschenksitte am Nikolaustag propagiert, um so das Interesse der Kinder auf Christus anstelle der Heiligenverehrung zu lenken. In römisch-katholischen Familien fand die Kinderbescherung weiterhin lange Zeit am Nikolaustag statt. Hinzu kamen alte und neue Bräuche verschiedener Herkunft, zum Beispiel Krippenspiele seit dem 11. Jahrhundert, zudem der geschmückte Weihnachtsbaum (16. Jahrhundert), der Adventskranz (1839) und der Weihnachtsmann (19. Jahrhundert). Dieser löste in Norddeutschland das Christkind und den Nikolaus als Gabenbringer für die Kinder ab. Viele Länder verbinden weitere eigene Bräuche mit Weihnachten. Der Besuch eines Gottesdienstes am Heiligen Abend ist auch bei Nicht-Kirchgängern verbreitet.\cite{kurz}


\section{Etymologie}

Der früheste Beleg für den Ausdruck „Weihnachten“, zusammengesetzt aus der adjektivischen Wendung ze wihen nahten, stammt aus der Predigtsammlung Speculum ecclesiae um 1170.\cite{reiser}

    „diu gnâde diu anegengete sih an dirre naht: von diu heizet si diu wîhe naht.“

    „Die Gnade (Gottes) kam zu uns in dieser Nacht: deshalb heißt diese nunmehr Weihnacht.“

Aus der gleichen Zeit (um 1190) stammt das Gedicht des bayerischen Dichters Spervogel:\cite{scala}

    „Er ist gewaltic unde starc, der ze wihen naht geborn wart: daz ist der heilige krist.“

    „Er ist gewaltig und stark, der zur Weihnacht geboren ward: Das ist der Heilige Christ.“


Schon früh wurde dagegen die Vermutung geäußert, dass der Name vorchristlichen Ursprungs sei: „das dieser heydnisch nam [Ostern] und standt nicht von Petro, sonder von den heyden in das christenthumb ist kommen, wie auch die fasznacht, weinnacht etc.“

Da die ersten Belege für das Wort aus dem 12. Jahrhundert stammen, nehmen viele Forscher an, dass der Begriff christlichen Ursprungs ist, vermutlich als Lehnübersetzung des lateinischen nox sancta aus den Gebeten der lateinischen Christmette unter Verwendung des volkssprachlichen Wortschatzes und ohne Bezug auf vorchristliche pagane Begriffsbildungen, wie ein Vergleich mit dem erhaltenen skandinavischen Begriff des Juls zeigt.

Martin Luther dachte an wiegen und bildete Wygenachten‚ „da wir das kindlein wiegen“.

Theodor Storm bildete aus dem Substantiv „Weihnachten“ dann das Verb weihnachten. In seinem Gedicht vom Knecht Ruprecht heißt es in den Anfangs- und Schlusszeilen:

Von drauß’ vom Walde komm ich her;
Ich muss euch sagen, es weihnachtet sehr.




