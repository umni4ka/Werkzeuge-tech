%!TEX root = thesis.tex

\chapter{Biblische Grundlagen}
\label{chapter-basics}

Die überwiegende Mehrheit in der historischen Jesusforschung kommt zu dem Schluss, dass Jesus von Nazaret geboren wurde und als Mensch in seiner Zeit gelebt hat. Seine Geburt in Bethlehem wird in zwei der vier Evangelien erzählt: Matthäus und Lukas stellen ihrem Evangelium jeweils unabhängig voneinander eine Kindheitsgeschichte voran. Die Erzählungen wollen aus nachösterlicher Sicht deutlich machen, dass Jesus Christus von Anfang an, bereits als neugeborenes Kind, der Sohn Gottes und der verheißene Messias war.\cite {rltl}

\section{Lukas und Matthäus}

Die heute geläufigere Darstellung stammt aus dem Lukasevangelium:

    „In jenen Tagen erließ Kaiser Augustus den Befehl, alle Bewohner des Reiches in Steuerlisten einzutragen. Dies geschah zum ersten Mal; damals war Quirinius Statthalter von Syrien. Da ging jeder in seine Stadt, um sich eintragen zu lassen.
    So zog auch Josef von der Stadt Nazaret in Galiläa hinauf nach Judäa in die Stadt Davids, die Betlehem heißt; denn er war aus dem Haus und Geschlecht Davids. Er wollte sich eintragen lassen mit Maria, seiner Verlobten, die ein Kind erwartete. Als sie dort waren, kam für Maria die Zeit ihrer Niederkunft, und sie gebar ihren Sohn, den Erstgeborenen. Sie wickelte ihn in Windeln und legte ihn in eine Krippe, weil in der Herberge kein Platz für sie war.“

– Lk 2,1–7 EU
Die Engel verkünden den Hirten die Geburt Christi, Darstellung aus dem Hortus Deliciarum der Herrad von Landsberg (12. Jahrhundert)

Es folgt bei Lukas die Verkündigung an die Hirten (Lk 2,8–20 EU) und die Darstellung Jesu im Tempel entsprechend jüdischer Vorschrift (Lk 2,21–40 EU). Vorausgegangen war die Verkündigung Jesu an Maria und parallel dazu die Verkündigung und die Geburt von Johannes dem Täufer (Lk 1,3–80 EU).

Das Matthäusevangelium spricht nach dem Stammbaum Jesu (Mt 1,1–17 EU) eher beiläufig von der Geburt Jesu, und zwar in Zusammenhang mit dem Zweifel Josefs an seiner Vaterschaft, dem ein Engel im Traum den Hinweis auf die Bedeutung des Kindes der Maria gab (Mt 1,18–25 EU). Es stellt die Verehrung des Neugeborenen durch die Sterndeuter dar (Mt 2,1–12 EU) und im Anschluss die Flucht nach Ägypten, den Kindermord des Herodes und die Rückkehr von Josef mit Maria und dem Kind nach Nazaret 

\begin{figure}
  \centering
  \pgfimage[width=.5\textwidth]{Christi}
  \caption[Abbildung 1. Die Engel verkünden den Hirten die Geburt Christi ]{Die Engel verkünden den Hirten die Geburt Christi, Darstellung aus dem Hortus Deliciarum der Herrad von Landsberg (12. Jahrhundert)}
  \label{Christi}
\end{figure}

\section{Theologische Aussage}

Das populäre „Maria legte das Kind in eine Krippe, weil in der Herberge kein Platz für sie war“ (Lk 2,7 EU) entspricht somit dem Satz des Johannesevangeliums „Er kam in sein Eigentum, aber die Seinen nahmen ihn nicht auf“ (Joh 1,11 EU) und der „Inkarnation“ und „Entäußerung“, dem „Den-Menschen-gleich-Werden“ der paulinischen Theologie (Phil 2,7 EU). Die Aussagen der Evangelien zur Geburt kennzeichnen die gesamte Sendung Jesu Christi als Gottes Heilstat zur Erlösung der Menschen durch seinen Sohn, von Jesu Geburt bis zu seiner Hinrichtung am Kreuz: „Schon in der Geburt ist (oder: wird) hier Jesus der Sohn Gottes“, betonen Matthäus und Lukas, indem sie ihrem Evangelium die weihnachtliche Vorgeschichte voranstellen



