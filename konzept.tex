%!TEX root = thesis.tex

\chapter{Kulturelle Aspekte}
\label{chapter-Kulturelle Aspekte}

Die früheste bekannte Darstellung der Geburt Jesu Christi stammt aus der Zeit um 320.Dort ist die Krippe der Form eines Altares angeglichen.

\section{Ikonographie}

\begin{figure}
  \centering
  \pgfimage[width=.5\textwidth]{bild}
  \caption[Abbildung 2.Buoninsegna ]{ Buoninsegna, Die Geburt Christi}
  \label{Buoninsegna}
\end{figure}

Die christliche Ikonographie entwickelte ihre Motive zunächst aus den Erzählungen des Matthäus- und Lukasevangelium sowie aus den apokryphen Kindheitsevangelien. Hinzu kamen viele Legendentexte verschiedener Herkunft. Seit den Darstellungen in den Katakomben im 3. Jahrhundert bis weit in die Renaissance wurde die Geburtsszene mit der Verkündigung an die Hirten und der Anbetung der Magier verbunden. Der Stall kommt im 4. Jahrhundert hinzu. Sehr früh schon thematisieren die Bilder die besondere Beziehung Jesu zu Maria, zum Beispiel das erste Bad oder die das Jesuskind stillende Mutter, wobei über Maria ein Stern steht (Domitilla- und Priscilla-Katakomben, spätes 3. Jahrhundert). Zu einem neuen Thema führte die Entdeckung der Geburtsgrotte durch die hl. Helena und die Erbauung der Geburtskirche durch Kaiser Konstantin. Schon seit dem frühen 4. Jahrhundert befinden sich Ochs und Esel auf den Bildern, die auf Jesaja 1,3 verweisen: „Der Ochs kennt seinen Besitzer, der Esel seine Krippe“. Sie und die Magier auf demselben Bild bedeuten, dass sowohl die höchsten als auch die niedrigsten Lebewesen das Kind anbeten. Auch symbolisierte der Ochs als reines Tier das jüdische Volk, das an das Gesetz gebunden ist, der Esel als unreines Tier die heidnischen Völker unter der Last des Heidentums. In den byzantinischen Darstellungen sind auch die beiden Hebammen Zelomi und Salome dargestellt, die in der christologischen Auseinandersetzung der damaligen Zeit die wirkliche menschliche Geburt Jesu betonen sollen. Die an der jungfräulichen Geburt Jesu zweifelnde Salome will diesen Umstand mit ihrer Hand untersuchen, die dann zur Strafe verdorrt. Die Berührung des Jesusknaben heilt sie wieder.[42] Dieses Thema ist im 5. und 6. Jahrhundert ein beliebtes Motiv der östlichen Kunst und ist auf der linken vorderen Ciboriumssäule (Ciborium ist ein Baldachin) von San Marco in Venedig, die aus Konstantinopel geraubt wurde, dargestellt.


\section{Musik}

Die weihnachtliche Kirchenmusik hat ihren Ursprung in der Ausgestaltung der drei Heiligen Messen, die an diesem Tag gefeiert werden dürfen (eigene Hymnen und Responsorien sind bereits seit frühchristlicher Zeit bekannt) sowie dem Gloria der Engel bei den Hirten auf dem Felde, von dem das Lukasevangelium 2,14 berichtet.
